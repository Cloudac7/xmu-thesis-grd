%%==================================================
%% chapter01.tex for BIT Master Thesis
%% modified by yang yating
%% version: 0.1
%% last update: Dec 25th, 2016
%%==================================================
\chapter{绪论}
\enchapter{Introduction}
\label{chap:intro}
\section{研究背景及选题意义}
\ensection{Research Background and Motivation}

近年来,随着人们生活水平的不断提高,人们越来越注重周围环境对身体健康的影响。作为服装是人们时时刻刻最贴近的环境,尤其是内衣,对人体健康有很大的影响。由于合时刻刻最贴近的环境,尤其是内衣,对人体健康有很大的影响。由于合成纤维的衣着舒适性、手感性,天然纤维的发展又成为人们关注的一大热点。\supercite{Takahashi1996Structure,Xia2002Analysis,Jiang1989,Mao2000Motion,Feng1998}

\section{国内外研究现状}
\ensection{Research Progress Overview in Home and Abroad}
%\label{sec:***} 可标注label

\section{本文主要内容和章节安排}
\ensection{Major Contents and Chapter Arrangement}

%\label{sec:features}

形状记忆聚合物(SMP)是继形状记忆合金后在80年代发展起来的一种新型形状记忆材料\cite{Jiang2005Size}。形状记忆高分子材料在常温范围内具有塑料的性质,即刚性、形状稳定恢复性;同时在一定温度下(所谓记忆温度下)具有橡胶的特性,主要表现为材料的可变形性和形变恢复性。即“记忆初始态-固定变形-恢复起始态”的循环。

固定相只有物理交联结构的聚氨酯称为热塑性SMPU,而有化学交联结构称为热固性SMPU。热塑性和热固性形状记忆聚氨酯的形状记忆原理示意图如图\ref{fig:diagram}所示

\begin{figure}
 \centering
 \includegraphics[width=0.75\textwidth]{figures/figure1}
 \caption{热塑性形状记忆聚氨酯的形状记忆机理示意图}\label{fig:diagram}
\end{figure}


\subsection{形状记忆聚氨酯的研究进展}
%\label{sec:requirements}
首例SMPU是日本Mitsubishi公司开发成功的……。

\subsection{水系聚氨酯及聚氨酯整理剂}

水系聚氨酯的形态对其流动性,成膜性及加工织物的性能有重要影响,一般分为三种类型\cite{Jiang2005Size} ,如表 \ref{tab:category}所示。

\begin{table}
  \centering
  \caption{水系聚氨酯分类} \label{tab:category}
  \begin{tabular*}{0.9\textwidth}{@{\extracolsep{\fill}}cccc}
  \toprule
    类别			&水溶型		&胶体分散型		&乳液型 \\
  \midrule
    状态			&溶解$\sim$胶束	&分散		&白浊 \\
    外观			&水溶型		&胶体分散型		&乳液型 \\
    粒径$/\mu m$	&$<0.001$		&$0.001-0.1$		&$>0.1$ \\
    重均分子量	&$1000\sim 10000$	&数千$\sim 20万$ &$>5000$ \\
  \bottomrule
  \end{tabular*}
\end{table}

由于它们对纤维织物的浸透性和亲和性不同,因此在纺织品染整加工中的用途也有差别,其中以水溶型和乳液型产品较为常用。另外,水系聚氨酯又有反应性和非反应性之分。虽然它们的共同特点是分子结构中不含异氰酸酯基,但前者是用封闭剂将异氰酸酯基暂时封闭,在纺织品整理时复出。相互交联反应形成三维网状结构而固着在织物表面。
……

然而索引扩散并不总是有效率的,它也会带来带宽开销。一方面,扩散更多的索引可以使搜索更快地返回,减少了搜索带宽开销;另一方面,由于P2P中结点和数据处于不断动态变化之中,当数据失效或更新时(如结点离线、删除或更新数据),数据的索引也相应失效,必须加以更新维护。因此,扩散更多的索引意味着维护开销的增加。于是在带宽开销方面,搜索开销与索引维护开销之间存在着折衷关系(trade-off)。与以往工作中仅考虑搜索开销不同,本章的模型中我们同时考虑搜索和维护两方面,给出了索引扩散方法对搜索整体性能的影响和数学关系。通过模型我们发现索引数量是决定宽松约束一般性搜索性能的至关重要的因素,采用最优索引分布可以很大程度上提高性能,降低系统开销。与一般认为的P2P无偏向性搜索难于扩展(non-scalable)恰恰相反,模型显示在最优的索引扩散策略下,基于无偏向性搜索具备很好的可扩展性,其结点负载和带宽开销随系统规模N(结点数)增长具有O( )的增长关系。这种平方根关系保证了对大规模P2P系统很好的适应性。

